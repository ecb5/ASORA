\message{ !name(SolutionManual.tex)}\documentclass[oneside]{book}

\usepackage{wrapfig}
\usepackage{ifpdf}
 \ifpdf
  \usepackage{hyperref}
\fi


\RequirePackage{amsmath}
\RequirePackage{amssymb}
\RequirePackage{amsthm}

\usepackage{graphicx}
\usepackage{color}
\usepackage{makeidx}

\def\halmos{\mbox{\raggedright\rule{0.1in}{0.1in}}}
\newcommand{\xqedhere}[2]{%
  \rlap{\hbox to#1{\hfil\llap{\ensuremath{#2}}}}}

\newcommand{\xqed}[1]{%
  \leavevmode\unskip\penalty9999 \hbox{}\nobreak\hfill
  \quad\hbox{\ensuremath{#1}}}

\def\IndexTheorem#1{\index{Theorems!Theorem~\ref{#1}}}
\def\IndexDefinition#1{\index{Definitions!Definition~\ref{#1}}}
\def\IndexCorollary#1{\index{Corollaries!Corollary~\ref{#1}}}
\def\IndexLemma#1{\index{Lemmas!Lemma~\ref{#1}}}


\def\abs#1{\left|#1\right|}
\def\divides#1#2{#1\kern.1em\left|#2\right.{}}
\def\notdivide#1#2{#1\kern-.2em\not|\,#2}

\newenvironment{ProofOutline}{\noindent{}{\bf Sketch of Proof:}}{\hfill{}QED?\\}
\newenvironment{hint}{\noindent{}\hfill\\{\rm\bf Hint: }}{}
\newenvironment{scrapwork}{\noindent{}\hfill\\{\bf{} SCRAPWORK}:
}{\bf\hfill\\ END OF SCRAPWORK} 
%\renewenvironment{proof}{\noindent{}{\bf Proof:\,}}{\hfill \halmos{}\\[2mm]}
\renewenvironment{proof}{\leftline{{\bf Proof:\,}}}{\hfill \halmos{}\\[2mm]}

%% Set up margin notes
\setlength{\marginparwidth}{1.2in}
\let\oldmarginpar\marginpar
\renewcommand\marginpar[1]{\-\oldmarginpar[\raggedleft\footnotesize #1]%
{\raggedright\footnotesize #1}}
%



\newenvironment{solution}[1]{{\color{red}{}\ \\\noindent{}\sc
    \underline{Solution:}\\}#1}{\hfill{}\\\color{red}{\sc \underline{End of Solution}}}%$\clubsuit$}
   


%% Odds and Ends
\def\imp{\ \Rightarrow\ }
\def\d#1{\thinspace{\rm d}#1}
\def\dfdx#1#2{\frac{\text{d}{#1}}{\text{d}{#2}}} 
\def\abs#1{\left|#1\right|}
\def\limit#1#2#3{{\displaystyle\lim_{#1\rightarrow #2}#3}}

\def\LabelProblem#1#2{\label{#1}\addcontentsline{toc}{subsection}{\hskip1cm Problem~\ref{#1}}}

\newcommand{\eps}{\varepsilon}
\newcommand{\unif}{\stackrel{unif}{\longrightarrow}}
\newcommand{\ptwise}{\stackrel{ptwise}{\longrightarrow}}

\newcommand{\CC}{\mathbb {C}}
\newcommand{\DD}{\mathbb {D}}
\newcommand{\RR}{\mathbb {R}}
\newcommand{\QQ}{\mathbb {Q}}
\newcommand{\NN}{\mathbb {N}}
\newcommand{\ZZ}{\mathbb {Z}}


\newtheorem{problem}{Problem}
\newtheorem{definition}{Definition}
\newtheorem{theorem}{Theorem}
\newtheorem{example}{Example}
\newtheorem{corollary}{Corollary}
\newtheorem{lemma}{Lemma}


\pagestyle{headings}{}
 \makeindex{}
\begin{document}

\message{ !name(SolutionManual.tex) !offset(1803) }
\begin{problem}
\LabelProblem{prob:rearrangement of alternating Harmonic series-diverges to infinity}{}
  Show that there is a rearrangement of
  $1-\frac{1}{2}+\frac{1}{3}-\frac{1}{4}+\cdots$ which diverges to
  $\infty$.\xqed{\lozenge}{}
  \begin{solution}{}
    Our proof will be by induction.
\vskip5mm{}
    \noindent{}{\bf \underline{Base case $(n=1):$}} Choose $O_1\in\NN$ such that
    $\exists$ $ k_1$ such that $$\sum_{i=1}^{k_1}\frac{1}{2i-1}> 2.$$

    Then $\left(\sum_{i=1}^{k_1}\frac{1}{2i-1}\right)-\frac12>1.$
\vskip5mm{}
    \noindent{}\underline{\bf{}Induction Assumption:} Assume that
    $k_1, k_2, \cdots, k_n$ has been constructed such
    that
    \begin{equation}
      \label{eq:1}
      \begin{split}
        \left[\left(\sum_{i=1}^{k_1}\frac{1}{2i-1}\right)-\frac12\right]
        &+ \left[\left(\sum_{i=k_1+1}^{k_2}\frac{1}{2i-1}\right) -
          \frac14\right] + \cdots \\
        &\quad\quad+ \left[\sum_{i=k_{n-1}+1}^{k_n}\frac{1}{2i-1} -\frac{1}{2n}\right] >
        n.
      \end{split}
      \end{equation}
      Observe that the summation in inequality~(\ref{eq:1}) is a
      rearrangement of all of the terms of the Alternating Harmonic
      Series up through $-\frac{1}{2n}.$
      
      Since $\sum_{i=1}^{\infty}\frac{1}{2i-1} = \infty$ we
    can find an integer $k+1$ such that
    $$\sum_{i=k_{n-1}+1}^{k_n}\frac{1}{2i-1} > 2.$$ And since
      $\frac{1}{2i-1} <1$ $\forall\  n\in\NN$ we see that
      $\left(\sum_{i=k_{n-1}+1}^{k_n}\frac{1}{2i-1}\right) -\frac{1}{2(k+1)}> 1.$
      Therefore 
    \begin{equation*}
      \begin{split}
        \left[\left(\sum_{i=1}^{k_1}\frac{1}{2i-1}\right)-\frac12\right]
        &+ \left[\left(\sum_{i=k_1+1}^{k_2}\frac{1}{2i-1}\right) -
          \frac14\right] \\
        &\quad+ \cdots\\
        &\quad\quad+ \left[\sum_{i=k_{n-1}+1}^{k_n}\frac{1}{2i-1} -\frac{1}{2n}\right]\\
        &\quad\quad\quad+\left[\sum_{i=k_n+1}^{k_{n+1}}\frac{1}{2i-1}
          -\frac{1}{2(n+1)}\right]>n+1.
      \end{split}
    \end{equation*}

      Observe that the summation in inequality~(\ref{eq:1}) is a
      rearrangement of all of the terms of the Alternating Harmonic
      Series up through $-\frac{1}{2(n+1)}.$
      Therefore, taking $k_0=0$ we see that  the rearrangement
      $$
      \sum_{n=k_0}^\infty \left(     \sum_{i=k_n+1}^{k_{n+1}}\frac{1}{2i-1} -\frac{1}{2n}\right)
      $$
      diverges to infinity.
\end{solution}
\newpage{}
\end{problem}

\message{ !name(SolutionManual.tex) !offset(6565) }

\end{document}
%%% Local Variables: 
%%% mode: latex
%%% TeX-master: t
%%% eval:(load-file (concat dropbox-location "/lisp/start-assessment.el"))
%%% problem-header: "\\begin{problem}"
%%% problem-tail: "\\end{problem}"
%%% End: 
