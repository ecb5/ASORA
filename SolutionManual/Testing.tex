\documentclass[oneside]{book}

\usepackage{wrapfig}
\usepackage{ifpdf}
 \ifpdf
  \usepackage{hyperref}
\fi


\RequirePackage{amsmath}
\RequirePackage{amssymb}
\RequirePackage{amsthm}

\usepackage{graphicx}
\usepackage{color}
\usepackage{makeidx}

\def\halmos{\mbox{\raggedright\rule{0.1in}{0.1in}}}
\newcommand{\xqedhere}[2]{%
  \rlap{\hbox to#1{\hfil\llap{\ensuremath{#2}}}}}

\newcommand{\xqed}[1]{%
  \leavevmode\unskip\penalty9999 \hbox{}\nobreak\hfill
  \quad\hbox{\ensuremath{#1}}}

\def\IndexTheorem#1{\index{Theorems!Theorem~\ref{#1}}}
\def\IndexDefinition#1{\index{Definitions!Definition~\ref{#1}}}
\def\IndexCorollary#1{\index{Corollaries!Corollary~\ref{#1}}}
\def\IndexLemma#1{\index{Lemmas!Lemma~\ref{#1}}}


\def\abs#1{\left|#1\right|}
\def\divides#1#2{#1\kern.1em\left|#2\right.{}}
\def\notdivide#1#2{#1\kern-.2em\not|\,#2}

\newenvironment{ProofOutline}{\noindent{}{\bf Sketch of Proof:}}{\hfill{}QED?\\}
\newenvironment{hint}{\noindent{}\hfill\\{\rm\bf Hint: }}{}
\newenvironment{scrapwork}{\noindent{}\hfill\\{\bf{} SCRAPWORK}:
}{\bf\hfill\\ END OF SCRAPWORK} 
%\renewenvironment{proof}{\noindent{}{\bf Proof:\,}}{\hfill \halmos{}\\[2mm]}
\renewenvironment{proof}{\leftline{{\bf Proof:\,}}}{\hfill \halmos{}\\[2mm]}

%% Set up margin notes
\setlength{\marginparwidth}{1.2in}
\let\oldmarginpar\marginpar
\renewcommand\marginpar[1]{\-\oldmarginpar[\raggedleft\footnotesize #1]%
{\raggedright\footnotesize #1}}
%



\newenvironment{solution}[1]{{\color{red}{}\ \\\noindent{}\sc
    \underline{Solution:}\\}#1}{\hfill{}\\\color{red}{\sc \underline{End of Solution}}}%$\clubsuit$}
   


%% Odds and Ends
\def\imp{\ \Rightarrow\ }
\def\d#1{\thinspace{\rm d}#1}
\def\dfdx#1#2{\frac{\text{d}{#1}}{\text{d}{#2}}} 
\def\abs#1{\left|#1\right|}
\def\limit#1#2#3{{\displaystyle\lim_{#1\rightarrow #2}#3}}

\def\LabelProblem#1#2{\label{#1}\addcontentsline{toc}{subsection}{\hskip1cm Problem~\ref{#1}}}

\newcommand{\eps}{\varepsilon}
\newcommand{\unif}{\stackrel{unif}{\longrightarrow}}
\newcommand{\ptwise}{\stackrel{ptwise}{\longrightarrow}}

\newcommand{\CC}{\mathbb {C}}
\newcommand{\DD}{\mathbb {D}}
\newcommand{\RR}{\mathbb {R}}
\newcommand{\QQ}{\mathbb {Q}}
\newcommand{\NN}{\mathbb {N}}
\newcommand{\ZZ}{\mathbb {Z}}


\newtheorem{problem}{Problem}
\newtheorem{definition}{Definition}
\newtheorem{theorem}{Theorem}
\newtheorem{example}{Example}
\newtheorem{corollary}{Corollary}
\newtheorem{lemma}{Lemma}

\begin{document}
\begin{proof}
Case 1: Suppose $b=1$. Let $\epsilon>0$ and let $N$ be any real number. If $n>N$, then 
\begin{align*}
\abs{b^{1/n}-1}=\abs{1-1}=0<\epsilon.\\
\end{align*}
Case 2: Suppose $b > 1$. Let $\epsilon >0$ and let $N=\frac{\log (b)}{\log (1+\epsilon)}$. If $n > N$, then 

\begin{align*}
\frac{1}{n}<\frac{1}{N}=\frac{\log (1+\epsilon)}{\log (b)} &\implies \frac{1}{n}\log (b)<\log (1+\epsilon)\\
&\implies \log (b^{1/n})<\log (1+\epsilon)\\
&\implies b^{1/n}<1+\epsilon \\
&\implies b^{1/n}-1=\abs{b^{1/n}-1}<\epsilon.\\
\end{align*}
Case 3.1: Suppose $0<b<1$. Let $0<\epsilon<1$ and let 
$N=\frac{\log (b)}{\log(1-\epsilon)}$. If $n > N$, then 

\begin{align*}
\frac{1}{n}<\frac{1}{N}=\frac{\log (1-\epsilon)}{\log (b)} &\implies \frac{1}{n}\log (b)>\log (1-\epsilon)\\
&\implies \log (b^{1/n})>\log (1-\epsilon)\\
&\implies b^{1/n}>1-\epsilon \\
&\implies b^{1/n}-1>-\epsilon\\
&\implies 1-b^{1/n}=\abs{b^{1/n}-1}<\epsilon.\\
\end{align*}
Case 3.2: Suppose $0<b<1$. Let $\epsilon \geq 1$ and let 
$N$ be any real number. If $n > N$, then 
\begin{align*}
b>0 &\implies b^{1/n}>0\\
&\implies -b^{1/n}<0\\
&\implies 1-b^{1/n}<1\leq \epsilon\\
&\implies \abs{b^{1/n}-1}<\epsilon.\\
\end{align*}
Therefore we can conclude that $\lim_{n \to \infty} b^{1/n}=1$.
\end{proof}
\end{document}

%%% Local Variables: 
%%% mode: latex
%%% TeX-master: t
%%% End: 
